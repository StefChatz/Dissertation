% $Id: chapter8.tex 1790 2010-09-28 16:46:40Z jabriffa $

\chapter {Conclusion}

In this section, I will look into how the overall  project progressed along with:

\begin{enumerate}
  \item What this project brings to an oversaturated market of social news aggregators.
  \item What parts of the project could be improved in order to scale and provide a more sophisticated solution.
  \item Things learned throughout the project.
\end{enumerate}

\section{Overview}

Social news and social media have redefined how people interact with the world, and over the last few years have dominated both the web and mobile market. Applications like Reddit, Pinterest and a whole lot more have become a place where people interact and socialise. Most of these websites operate in their niche category and require the user to adhere to guidelines such as only posting photos, or following content specific subreddits (in the case of Reddit). These websites rely on a select few moderators or administrators that dictate what the content of each category will be. This project provides a solution where all the content is "out there" and its up to the users to decide which type of information they see with not much interaction. This is achieved thanks to the website taking care of sorting the posts and provide content regulation, while the users are able to customise the algorithm.

\section{Potential Improvements}

Although the project achieves the goals and objectives, it set out to achieve. It is not to say, however, that there could not be any improvements. As mentioned before, the implementation of a NN that monitors the interactions of the user with the website components could prove beneficial since people's interests tend to change over time. Nevertheless, the use of an AI would require much computational power and even more testing in order to train the model and provide accurate predictions on the user's interests.

\section{Personal Gains}

While developing this project, I faced many challenges, both in the development process and real-world issues. This project allowed me to venture more into my passion for Web Design and provide a solution to a problem. Furthermore, I was given the opportunity with this project to explore new languages and frameworks that gave me a broader view of what I would like to pursue professionally. The most important lesson I learned during the development, is how the correct and consistent use of the Software Development Lifecycle (SDLC) and even more is the use of the agile methodology leads to the development of high-quality software. All in all, this project was an eye-opening experience in terms of the acquirement of new skills but also, in testing my capabilities and what I can achieve when developing enjoyable software.
